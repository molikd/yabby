% chapter03.tex

 %%%%%%%%%%%%%%%%%%%%%%%%%%%%%%%%%%%%%%%%%%%%%%%%%%%%%%%%%%%%%%%%%%%%%%%%%%%%%
 %                                                                           %
 %    YABBY documentation                                                    %
 %    Copyright (C) 2007 Vladimir Likic                                      %
 %                                                                           %
 %    The files in this directory provided under the Creative Commons        %
 %    Attribution-NonCommercial-NoDerivs 2.1 Australia license               %
 %    http://creativecommons.org/licenses/by-nc-nd/2.1/au/                   %
 %    See the file license.txt                                               %
 %                                                                           %
 %%%%%%%%%%%%%%%%%%%%%%%%%%%%%%%%%%%%%%%%%%%%%%%%%%%%%%%%%%%%%%%%%%%%%%%%%%%%%

\setcounter{section}{0}

\chapter{Command reference}

\section{General commands}

General commands are related to the basic Yabby functions, such as
control of the workspace, and are not related to any specific area
of application. 


% ### NEW COMMAND ###

\subsection[delete]{ \fbox{\tt delete} }

\index{\tt delete}

% Concise explanation

Deletes objects from the workspace.

% Command description

\begin{description}

% 1. Usage

\item{Usage:}

{\tt delete OBJECT.PROPERTY}

% 2. Options

\item{Options:}
\begin{description}
{\em None}
\end{description}

% 3. Notes

\item{Notes:}
\begin{enumerate}
\item If PROPERTY equals '*' (no quotes) all properties of OBJECT will
be deleted.
\end{enumerate}

% 4. Examples 

\item{Examples:}
\begin{enumerate}

\item
\begin{verbatim}
yabby> delete sp.seq

 [ 'sp.seq' deleted ]
\end{verbatim}

\end{enumerate}

% 5. Requirements 

\item{Requirements:} {\em None}

% 6. Objects 

\item{Creates objects:} {\em None}

% 7. System scripts

\item{System scripts:} {\em None}

\end{description}

% ### END OF NEW COMMAND ####

% ### NEW COMMAND ###

\subsection[dump]{ \fbox{\tt dump} }

\index{\tt dump}

% Concise explanation

Dumps the current workspace to a file, which later can be restored
with 'restore'.

% Command description

\begin{description}

% 1. Usage

\item{Usage:}

{\tt dump SESSION\_NAME}

% 2. Options

\item{Options:}
\begin{description}
{\em None}
\end{description}

% 3. Notes

\item{Notes:}
\begin{enumerate}
\item This command saves the current Yabby session in the file
 SESSION\_NAME.tar.gz in the current directory.
\item Currently works only on system that have GNU tar command.
\end{enumerate}

% 4. Examples 

\item{Examples:}
\begin{enumerate}

\item
\begin{verbatim}
yabby> dump tmpsession

 Yabby session archived as 'tmpsession.tar.gz'
\end{verbatim}

\end{enumerate}

% 5. Requirements 

\item{Requirements:} {\em None}

% 6. Objects 

\item{Creates objects:} {\em None}

% 7. System scripts

\item{System scripts:} {\em None}

\end{description}

% ### END OF NEW COMMAND ####


% ### NEW COMMAND ###

\subsection[flush]{ \fbox{\tt flush} }

\index{\tt flush}

% Concise explanation

Deletes everything from the workspace.

% Command description

\begin{description}

% 1. Usage

\item{Usage:}

{\tt flush}

% 2. Options

\item{Options:}
\begin{description}
{\em None}
\end{description}

% 3. Notes

\item{Notes:}
\begin{enumerate}
{\em None}
\end{enumerate}

% 4. Examples 

\item{Examples:}
\begin{enumerate}

\item
\begin{verbatim}
yabby> flush

 [ workspace flushed ]
\end{verbatim}

\end{enumerate}

% 5. Requirements 

\item{Requirements:} {\em None}

% 6. Objects 

\item{Creates objects:} {\em None}

% 7. System scripts

\item{System scripts:} {\em None}

\end{description}

% ### END OF NEW COMMAND ####


% ### NEW COMMAND ###

\subsection[license]{ \fbox{\tt license} }

\index{\tt license}

% Concise explanation

Prints the Yabby license.

% Command description

\begin{description}

% 1. Usage

\item{Usage:}

{\tt license}

% 2. Options

\item{Options:}
\begin{description}
{\em None}
\end{description}

% 3. Notes

\item{Notes:}
\begin{enumerate}
{\em None}
\end{enumerate}

% 4. Examples 

\item{Examples:}
\begin{enumerate}

\item
\begin{verbatim}
yabby> license


                    GNU GENERAL PUBLIC LICENSE
                       Version 2, June 1991

 Copyright (C) 1989, 1991 Free Software Foundation, Inc.
                          675 Mass Ave, Cambridge, MA 02139, USA
 Everyone is permitted to copy and distribute verbatim copies
 of this license document, but changing it is not allowed.

                            Preamble

  The licenses for most software are designed to take away your
freedom to share and change it.  By contrast, the GNU General Public
....
\end{verbatim}

\end{enumerate}

% 5. Requirements 

\item{Requirements:} {\em None}

% 6. Objects 

\item{Creates objects:} {\em None}

% 7. System scripts

\item{System scripts:} {\em None}

\end{description}

% ### END OF NEW COMMAND ####

% ### NEW COMMAND ###

\subsection[help]{ \fbox{\tt help} }

\index{\tt help}

% Concise explanation

Prints help pages.

% Command description

\begin{description}

% 1. Usage

\item{Usage:}

{\tt help [ COMMAND ]}

% 2. Options

\item{Options:}
\begin{description}
{\em None}
\end{description}

% 3. Notes

\item{Notes:}
\begin{enumerate}
\item If no argument is given the list of all available commands will
be printed. If a command name is given the help about a particular
command will be printed. 
\end{enumerate}

% 4. Examples 

\item{Examples:}
\begin{enumerate}

\item
\begin{verbatim}
yabby> help quit

 Exits Yabby

\end{verbatim}

\end{enumerate}

% 5. Requirements 

\item{Requirements:} {\em None}

% 6. Objects 

\item{Creates objects:} {\em None}

% 7. System scripts

\item{System scripts:} {\em None}

\end{description}

% ### END OF NEW COMMAND ####


% ### NEW COMMAND ###

\subsection[print]{ \fbox{\tt print} }

\index{\tt print}

% Concise explanation

Prints the an object on the terminal screen or to a file.

% Command description

\begin{description}

% 1. Usage

\item{Usage:}

{\tt print [ options ] OBJ\_NAME.PROPERTY}

% 2. Options

\item{Options (general):}
\begin{description}
\item -f FILE\_NAME -- print to a file instead on the terminal screen.
\end{description}

\item{Options (seq objects only):}
\begin{description}
\item -l -- print the sequence as the three-letter code.
\item -c -- print the sequences as the CSV table.
\item -n N -- print N residues per line (both when printing one- and
 three-letter codes).
\item -t N -- truncate each sequence at N letters when writing
\end{description}

% 3. Notes

\item{Notes:}
\begin{enumerate}
\item Currently supported objects for printing are 'seq', 'motif',
and 'blastg'. 
\end{enumerate}

% 4. Examples 

\item{Examples:}
\begin{enumerate}

\item Print the sequence object tom20.seq
\begin{verbatim}
yabby> print tom20.seq

>A.thaliana [ A.thaliana ]
MDTETEFDRILLFEQIRQDAENTYKSNPLDADNLTRWGGVLLELSQFHSISDAKQMIQEA
ITKFEEALLIDPKKDEAVWCIGNAYTSFAFLTPDETEAKHNFDLATQFFQQAVDEQPDNT
HYLKSLEMTAKAPQLHAEAYKQGLGSQPMGRVEAPAPPSSKAVKNKKSSDAKYDAMGWVI
LAIGVVAWISFAKANVPVSPPR
>O.sativa [ O.sativa ]
MDMGAMSDPERMFFFDLACQNAKVTYEQNPHDADNLARWGGALLELSQMRNGPESLKCLE
DAESKLEEALKIDPMKADALWCLGNAQTSHGFFTSDTVKANEFFEKATQCFQKAVDVEPA
NDLYRKSLDLSSKAPELHMEIHRQMASQASQAASSTSNTRQSRKKKKDSDFWYDVFGWVV
LGVGMVVWVGLAKSNAPPQAPR
>L.esculentum [ L.esculentum ]
MDMQSDFDRLLFFEHARKTAETTYATDPLDAENLTRWAGALLELSQFQSVSESKKMISDA
ISKLEEALEVNPQKHDAIWCLGNAYTSHGFLNPDEDEAKIFFDKAAQCFQQAVDADPENE
LYQKSFEVSSKTSELHAQIHKQGPLQQAMGPGPSTTTSSTKGAKKKSSDLKYDVFGWVIL
AVGLVAWIGFAKSNMPXPAHPLPR
>S.tuberosum [ S.tuberosum ]
MEMQSEFDRLLFFEHARKSAETTYAQNPLDADNLTRWGGALLELSQFQPVAESKQMISDA
TSKLEEALTVNPEKHDALWCLGNAHTSHVFLTPDMDEAKVYFEKATQCFQQAFDADPSND
LYRKSLEVTAKAPELHMEIHRHGPMQQTMAAEPSTSTSTKSSKKTKSSDLKYDIFGWVIL
AVGIVAWVGFAKSNMPPPPPPPPQ
>P.taeda [ P.taeda ]
MEEMAIPQSDFDRLLFFEGARKAAENTYAVNPEDADNLTRWGGALLELSQFQQGPDCVKM
VKDAVSKLEEALKISPNKHDTLWCLGNAHTSHAFLIPEHEVAKIYFKMASKYFQQAVEQD
PTNELYRKSLELTEKAPELHLEVHKQILNPQSVAAGSSTVSNLKGSRKKKSSDLKYDIMG
WIVLAVGIAAWVGMAKSHVPPPPML
>G.max [ G.max ]
MDLQQSEFDRLLFFEHARKAAEAEYEKNPLDADNLTRWGGALLELSQFQSFPESKKMTQE
AVSKLEEALAVNPKKHDTLWCLGNAHTSQAFLIPDQEEAKVYFDKAAVYFQQAVDEDPSN
ELYRKSLEVAAKAPELHVEIHKQGFGQQQQAAATAGSSTSASTNTQKKKKSSDLKYDIFG
WIILAVGIVAWVGFAKSNLPPPPPPPPR
>Z.mays [ Z.mays ]
MEMGGMSDAERLFFFEMACKNSEVAYEQNPNDADNLTRWGGALLELSQVRTGPDSLKLLE
DAEAKLEEALQIDPNKSDALWCLGNAQTSHGFFTPDNAIANEFFTKATGCFQKAVDVEPA
NELYRKSLDLSMKAPELHLEIQRQMVSQAATQASSASNPRQSRKKKDNDFWYDVCGWVIL
GAGIVAWVGLARASMPPPTPPAR
\end{verbatim}

\item Print the sequence object tom20.seq to a file 'tom20.txt' 
\begin{verbatim}
yabby> print -f tom20.txt tom20.seq

 'tom20.seq' written to the file 'tom20.txt'
\end{verbatim}

\end{enumerate}

% 5. Requirements 

\item{Requirements:} The object to be printed must exist in the
workspace.

% 6. Objects 

\item{Creates objects:} {\em None}

% 7. System scripts

\item{System scripts:} \_print\_seq.pl, \_print\_motif.pl, \_print\_blastg.pl

\end{description}

% ### END OF NEW COMMAND ####

% ### NEW COMMAND ###

\subsection[restore]{ \fbox{\tt restore} }

\index{\tt restore}

% Concise explanation

Restores Yabby session saved with the command 'dump'

% Command description

\begin{description}

% 1. Usage

\item{Usage:}

{\tt restore SESSION\_NAME}

Where SESSION\_NAME is the archive name used when dumping
the session.

% 2. Options

\item{Options:}
\begin{description}
{\em None}
\end{description}

% 3. Notes

\item{Notes:}
\begin{enumerate}
\item The session will sored to a file named SESSION\_NAME.tar.gz.
\item Relies on GNU gar and gzip commands. These must be in the
 executable path.
\end{enumerate}

% 4. Examples 

\item{Examples:}
\begin{enumerate}

\item
\begin{verbatim}
yabby> restore tmpsession

 Yabby session 'tmpsession' restored
\end{verbatim}

\end{enumerate}

% 5. Requirements 

\item{Requirements:} {\em None}

% 6. Objects 

\item{Creates objects:} {\em None}

% 7. System scripts

\item{System scripts:} {\em None}

\end{description}

% ### END OF NEW COMMAND ####


% ### NEW COMMAND ###

\subsection[what]{ \fbox{\tt what} }

\index{\tt what}

% Concise explanation

Shows objects currently in the workspace.

% Command description

\begin{description}

% 1. Usage

\item{Usage:}

{\tt what}

% 2. Options

\item{Options:}
\begin{description}
{\em None}
\end{description}

% 3. Notes

\item{Notes:}
\begin{enumerate}
{\em None}
\end{enumerate}

% 4. Examples 

\item{Examples:}
\begin{enumerate}

\item
\begin{verbatim}
yabby> what

    objects        properties
  ------------------------------
    tom20          seq           

\end{verbatim}

\end{enumerate}

% 5. Requirements 

\item{Requirements:} {\em None}

% 6. Objects 

\item{Creates objects:} {\em None}

% 7. System scripts

\item{System scripts:} {\em None}

\end{description}

% ### END OF NEW COMMAND ####

% ##############################################################
\section{Sequence commands}
% ##############################################################

% ### NEW COMMAND ###

\subsection[seq\_comment]{ \fbox{\tt seq\_comment} }

\index{\tt seq\_comment}

% Concise explanation

Modifies sequence comments to add a unique number.

% Command description

\begin{description}

% 1. Usage

\item{Usage:}

{\tt seq\_comment OBJ\_NAME}

Where OBJ\_NAME is the internal Yabby name for the sequence(s).

% 2. Options

\item{Options:}
\begin{description}
{\em None}
\end{description}

% 3. Notes

\item{Notes:}
\begin{enumerate}
{\em None}
\end{enumerate}

% 4. Examples 

\item{Examples:}
\begin{enumerate}

\item
\begin{verbatim}
yabby> seq_comment tom20

 Comments modified.

\end{verbatim}

\end{enumerate}

% 5. Requirements 

\item{Requirements:} {\em None}

% 6. Objects 

\item{Creates objects:} {\em None}

% 7. System scripts

\item{System scripts:} {\em None}

\end{description}

% ### END OF NEW COMMAND ####


% ### NEW COMMAND ###

\subsection[seq\_compl]{ \fbox{\tt seq\_compl} }

\index{\tt seq\_compl}

% Concise explanation

Calculates DNA sequence complement.

% Command description

\begin{description}

% 1. Usage

\item{Usage:}

{\tt seq\_compl [ options ] OBJ\_NAME OBJ\_NAME\_NEW}

Where OBJ\_NAME is the name of an existing DNA sequence object,
the complement sequence will be saved under the name OBJ\_NAME\_NEW.


% 2. Options

\item{Options:}
\begin{description}
\item -r -- reverse complement
\end{description}

% 3. Notes

\item{Notes:} {\em None}

% 4. Examples 

\item{Examples:}
\begin{enumerate}

\item
\begin{verbatim}
yabby> seq_compl dna dna_compl

 'dna' contains 2 sequence(s)
 Working on 'chr01'
 Working on 'chr02'

\end{verbatim}

\end{enumerate}

% 5. Requirements 

\item{Requirements:} {\em None}

% 6. Objects 

\item{Creates objects:} another seq

% 7. System scripts

\item{System scripts:} {\em None}

\end{description}

% ### END OF NEW COMMAND ####


% ### NEW COMMAND ###

\subsection[seq\_fetch]{ \fbox{\tt seq\_fetch} }

\index{\tt seq\_fetch}

% Concise explanation

Fetch a sequence from GenBank.

% Command description

\begin{description}

% 1. Usage

\item{Usage:}

{\tt seq\_fetch ACC\_NUM OBJ\_NAME}

Fetch a sequence from GenBank using sequence accession number,
and save the sequence under the name OBJ\_NAME. This command
is mainly a demonstration of how to use BioPerl library within
Yabby.

% 2. Options

\item{Options:} {\em None}

% 3. Notes

\item{Notes:}
\begin{enumerate}
\item This command requires Bioperl's Bio::DB module.
\end{enumerate}

% 4. Examples 

\item{Examples:}
\begin{enumerate}

\item
\begin{verbatim}
yabby> seq_fetch J00522 mig

 Saving 'J00522' as 'mig'

yabby> print mig.seq

>gi|195052|gb|J00522.1|MUSIGHBA1 [ Mouse Ig active H-chain V-region from MOPC21, subgroup VH-II, mRNA ]
AGGCTCAATTTAGTTTTCCTTGTCCTTATTTTAAAAGGTGTCCAGTGTGATGTGCAGCTG
GTGGAGTCTGGGGGAGGCTTAGTGCAGCCTGGAGGGTCCCGGAAACTCTCCTGTGCAGCC
TCTGGATTCACTTTCAGTAGCTTTGGAATGCACTGGGTTCGTCAGGCTCCAGAGAAGGGG
CTGGAGTGGGTCGCATACATTAGTAGTGGCAGTAGTACCCTCCACTATGCAGACACAGTG
AAGGGCCGATTCACCATCTCAAGAGACAATCCCAAGAACACCCTGTTCCTGCAAATGACC
AGTCTAAGGTCTGAGGACACGGCCATGTATTACTGTGCAAGATGGGGTAACTACCCTTAC
TATGCTATGGACTACTGGGGTCAAGGAACCTCAGTCACCGTCTCCTCA
\end{verbatim}

\end{enumerate}

% 5. Requirements 

\item{Requirements:} {\em None}

% 6. Objects 

\item{Creates objects:} seq

% 7. System scripts

\item{System scripts:} {\em None}

\end{description}

% ### END OF NEW COMMAND ####


% ### NEW COMMAND ###

\subsection[seq\_info]{ \fbox{\tt seq\_info} }

\index{\tt seq\_info}

% Concise explanation

Prints information about the sequence object.

% Command description

\begin{description}

% 1. Usage

\item{Usage:}

{\tt seq\_info [ options ] OBJ\_NAME}

Where OBJ\_NAME is the name of an existing sequence object.

% 2. Options

\item{Options:}
\begin{description}
\item -l -- long output.  When multiple sequences are present,
            print the number of residues for each sequence.
            By default, only a short summary is printed.
\item -n -- print only the number of residues in each sequence
\end{description}

% 3. Notes

\item{Notes:} {\em None}

% 4. Examples 

\item{Examples:}
\begin{enumerate}

\item
\begin{verbatim}
yabby> seq_info Tim10

 'Tim10' contains 13 sequence(s)
   min number of residues: 75 (sequence 'Plasmodium')
   max number of residues: 115 (sequence 'Ciona')
mouse
Schistosoma
Dictyostelium
Schizosaccharomyces
Saccharomyces
Neurospora
Ciona
Xenopus
Danio
Anopheles
Caenorhabditis
Plasmodium
Arabidopsis

yabby> seq_info -l Tim10

 'Tim10' contains 13 sequence(s)
  1 -> mouse, 90 residues
  2 -> Schistosoma, 88 residues
  3 -> Dictyostelium, 88 residues
  4 -> Schizosaccharomyces, 89 residues
  5 -> Saccharomyces, 93 residues
  6 -> Neurospora, 90 residues
  7 -> Ciona, 115 residues
  8 -> Xenopus, 93 residues
  9 -> Danio, 88 residues
 10 -> Anopheles, 92 residues
 11 -> Caenorhabditis, 86 residues
 12 -> Plasmodium, 75 residues
 13 -> Arabidopsis, 83 residues

yabby> seq_info -n Tim10

 'Tim10' contains 13 sequence(s)
90
88
88
89
93
90
115
93
88
92
86
75
83
\end{verbatim}

\end{enumerate}

% 5. Requirements 

\item{Requirements:} {\em None}

% 6. Objects 

\item{Creates objects:} {\em None}

% 7. System scripts

\item{System scripts:} {\em None}

\end{description}

% ### END OF NEW COMMAND ####


% ### NEW COMMAND ###

\subsection[seq\_load]{ \fbox{\tt seq\_load} }

\index{\tt seq\_load}

% Concise explanation

Loads sequence(s) from the database file.

% Command description

\begin{description}

% 1. Usage

\item{Usage:}

{\tt seq\_load [ options ] DBA\_FILE OBJ\_NAME}

Where DBA\_FILE is the name of the database file. OBJ\_NAME is
the internal Yabby name for the sequence(s).

% 2. Options

\item{Options:}
\begin{description}
\item -a -- append sequences to an already existing sequence
object OBJ\_NAME
\item -f -- the file format is FASTA (default)
\item -b -- the file format is BLOCKS
\end{description}

% 3. Notes

\item{Notes:}
\begin{enumerate}
\item Only BLOCKS format written by MEME \cite{meme} was tested.
\end{enumerate}

% 4. Examples 

\item{Examples:}
\begin{enumerate}

\item
\begin{verbatim}
yabby> seq_load tom20.fas tom20

 Reading the file 'tom20.fas' ..
 7 sequence(s) found.
\end{verbatim}

\end{enumerate}

% 5. Requirements 

\item{Requirements:} {\em None}

% 6. Objects 

\item{Creates objects:} seq

% 7. System scripts

\item{System scripts:} {\em None}

\end{description}

% ### END OF NEW COMMAND ####

% ### NEW COMMAND ###

\subsection[seq\_op]{ \fbox{\tt seq\_op} }

\index{\tt seq\_op}

% Concise explanation

Calculates union/intersection/difference of two sequence objects
by using the sequence IDs.

% Command description

\begin{description}

% 1. Usage

\item{Usage:}

{\tt seq\_op [ options ] SEQ1\_OBJ SEQ2\_OBJ RES\_OBJ}

Calculates union/intersection/difference of two sequence objects
SEQ1\_OBJ and SEQ2\_OBJ, and stores the result as the sequence
object RES\_OBJ.

% 2. Options

\item{Options:}
\begin{description}
\item -u -- calculate the union (default)
\item -i -- calculate the intersection
\item -d -- calculate the difference
\end{description}

% 3. Notes

\item{Notes:}
\begin{enumerate}
\item The calculation will fail if there are duplicate sequences in
      one set. For example, if two sets of sequences have no sequence
      in common, but one set of sequences contains two copies of the
      sequence 'F36.5845', the intersection of the two sets will contain
      this sequence.
\end{enumerate}

% 4. Examples 

\item{Examples:}
\begin{enumerate}

\item
\begin{verbatim}
yabby> seq_op -i Tim10 Tim10-5 Tim10int

 Found 13 sequence(s) in 'Tim10'
 Found 8 sequence(s) in 'Tim10-5'
  INTERSECTION contains 8 sequence(s)
  DIFFERENCE contains 5 sequence(s)
  UNION contains 13 sequence(s)
 [ Saving INTERSECTION as 'Tim10int' ]

\end{verbatim}

\end{enumerate}

% 5. Requirements 

\item{Requirements:} {\em None}

% 6. Objects 

\item{Creates objects:} seq

% 7. System scripts

\item{System scripts:} {\em None}

\end{description}

% ### END OF NEW COMMAND ####


% ### NEW COMMAND ###

\subsection[seq\_pattern]{ \fbox{\tt seq\_pattern} }

\index{\tt seq\_pattern}

% Concise explanation

Searches for pattern in a sequence object.

% Command description

\begin{description}

% 1. Usage

\item{Usage:}

{\tt seq\_pattern [ options ] PATTERN OBJ\_NAME}

Searches for pattern PATTERN in sequences OBJ\_NAME.
Where OBJ\_NAME is the name of an existing sequence object.

% 2. Options

\item{Options:}
\begin{description}
\item -c -- match the comment not the sequence
\item -s NAME -- save the matching sequences under the name NAME 
                 should this read another OBJ\_NAME e.g. OBJ\_NAME\_NEW
\end{description}

% 3. Notes

\item{Notes:} {\em None}

% 4. Examples 

\item{Examples:}
\begin{enumerate}

\item
\begin{verbatim}
bug here 
yabby> seq_pattern -c habditis Tim10

1 1
 'habditis' matches in 'Caenorhabditis'
 13 sequences examined, 1 matches found

yabby> seq_pattern NSTVVELLG Tim10

Use of uninitialized value in concatenation (.) or string at seq_pattern.pl line 42
0
 'NSTVVELLG' matches in 'Plasmodium'
 13 sequences examined, 1 matches found

seq_pattern -s seqinTim10 NSTVVELLG Tim10
Use of uninitialized value in concatenation (.) or string at seq_pattern.pl line 42
0
 'NSTVVELLG' matches in 'Plasmodium'
 13 sequences examined, 1 matches found

\end{verbatim}

\end{enumerate}

% 5. Requirements 

\item{Requirements:} {\em None}

% 6. Objects 

\item{Creates objects:} with option s, creates another seq

% 7. System scripts

\item{System scripts:} {\em None}

\end{description}

% ### END OF NEW COMMAND ####


% ### NEW COMMAND ###

\subsection[seq\_pick]{ \fbox{\tt seq\_pick} }

\index{\tt seq\_pick}

% Concise explanation

Extracts a subset of sequences from the sequence object.

% Command description

\begin{description}

% 1. Usage

\item{Usage:}

{\tt seq\_pick [ options ] OBJ\_NAME\_NEW OBJ\_NAME}

Where OBJ\_NAME is the name of an existing sequence object
and OBJ\_NAME\_NEW is the name of the sequence object to be created.

% 2. Options

\item{Options:}
\begin{description}
\item -n RANGE -- extract the sequence by number range.
                  The range parameter RANGE can be a single integer, in which
                  the sequence with this sequence number will be extracted.
                  Alternatively, RANGE can contain two integers separated
                  by a colon such as N:M.  In this case the range of sequences
                  between N and M will be extracted (inclusive).
\item -q SEQID -- pick a sequence with the sequence ID SEQID
\item -l MIN:MAX -- pick sequences whose length is between MIN and MAX
\end{description}

% 3. Notes

\item{Notes:} {\em None}

% 4. Examples 

\item{Examples:}
\begin{enumerate}

\item
\begin{verbatim}
yabby> seq_pick -q Arabidopsis Tim10Arabidopsis Tim10

 Fetching the sequence 'Arabidopsis'
 Saving the extracted sequence as 'Tim10Arabidopsis

yabby> seq_pick -n 1:2 Tim10-12 Tim10

 Saving the extracted sequence as 'Tim10-12'

yabby> seq_pick -l 115:130 Tim10l Tim10

 1 sequences matched
 Saving the extracted sequence as 'Tim10l'

\end{verbatim}

\end{enumerate}

% 5. Requirements 

\item{Requirements:} {\em None}

% 6. Objects 

\item{Creates objects:} seq

% 7. System scripts

\item{System scripts:} {\em None}

\end{description}

% ### END OF NEW COMMAND ####

% ### NEW COMMAND ###

\subsection[seq\_strip]{ \fbox{\tt seq\_strip} }

\index{\tt seq\_strip}

% Concise explanation

Strips a portion of a sequence.

% Command description

\begin{description}

% 1. Usage

\item{Usage:}

{\tt seq\_strip begin:end OBJ\_NAME OBJ\_NAME\_NEW}

Where OBJ\_NAME is the name of an existing sequence object, and
begin:end are the first and last residue to strip (inclusive).
The resulting object will be saved under the name OBJ\_NAME\_NEW.

If more than one sequence is present in the sequence object,
all will be stripped and saved under the new name.

In stripped sequences, IDs are set to ORIGINALID\_begin:end.

% 2. Options

\item{Options:} {\em None}

% 3. Notes

\item{Notes:} {\em None}

% 4. Examples 

\item{Examples:}
\begin{enumerate}

\item
\begin{verbatim}
yabby> seq_strip 2:10 Tim10 Tim10stripped

 'Tim10' contains 13 sequence(s)
 stripping 'mouse'
 stripping 'Schistosoma'
 stripping 'Dictyostelium'
 stripping 'Schizosaccharomyces'
 stripping 'Saccharomyces'
 stripping 'Neurospora'
 stripping 'Ciona'
 stripping 'Xenopus'
 stripping 'Danio'
 stripping 'Anopheles'
 stripping 'Caenorhabditis'
 stripping 'Plasmodium'
 stripping 'Arabidopsis'

\end{verbatim}

\end{enumerate}

% 5. Requirements 

\item{Requirements:} {\em None}

% 6. Objects 

\item{Creates objects:} seq

% 7. System scripts

\item{System scripts:} {\em None}

\end{description}

% ### END OF NEW COMMAND ####

