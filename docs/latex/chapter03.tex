% chapter03.tex

 %%%%%%%%%%%%%%%%%%%%%%%%%%%%%%%%%%%%%%%%%%%%%%%%%%%%%%%%%%%%%%%%%%%%%%%%%%%%%
 %                                                                           %
 %    YABBY documentation                                                    %
 %    Copyright (C) 2007 Vladimir Likic                                      %
 %                                                                           %
 %    The files in this directory provided under the Creative Commons        %
 %    Attribution-NonCommercial-NoDerivs 2.1 Australia license               %
 %    http://creativecommons.org/licenses/by-nc-nd/2.1/au/                   %
 %    See the file license.txt                                               %
 %                                                                           %
 %%%%%%%%%%%%%%%%%%%%%%%%%%%%%%%%%%%%%%%%%%%%%%%%%%%%%%%%%%%%%%%%%%%%%%%%%%%%%

\setcounter{section}{0}

\chapter{Command reference}

\section{General commands}

General commands are related to the basic Yabby functions, such as
control of the workspace, and are not related to any specific area
of application. 


% ### NEW COMMAND ###

\subsection[delete]{ \fbox{\tt delete} }

\index{\tt delete}

% Concise explanation

Deletes objects from the workspace.

% Command description

\begin{description}

% 1. Usage

\item{Usage:}

{\tt delete OBJECT.PROPERTY}

% 2. Options

\item{Options:}
\begin{description}
{\em None}
\end{description}

% 3. Notes

\item{Notes:}
\begin{enumerate}
\item If PROPERTY equals '*' (no quotes) all properties of OBJECT will
be deleted.
\end{enumerate}

% 4. Examples 

\item{Examples:}
\begin{enumerate}

\item
\begin{verbatim}
yabby> delete sp.seq

 [ 'sp.seq' deleted ]
\end{verbatim}

\end{enumerate}

% 5. Requirements 

\item{Requirements:} {\em None}

% 6. Objects 

\item{Creates objects:} {\em None}

% 7. System scripts

\item{System scripts:} {\em None}

\end{description}

% ### END OF NEW COMMAND ####

% ### NEW COMMAND ###

\subsection[dump]{ \fbox{\tt dump} }

\index{\tt dump}

% Concise explanation

Dumps the current workspace to a file, which later can be restored
with 'restore'.

% Command description

\begin{description}

% 1. Usage

\item{Usage:}

{\tt dump SESSION\_NAME}

% 2. Options

\item{Options:}
\begin{description}
{\em None}
\end{description}

% 3. Notes

\item{Notes:}
\begin{enumerate}
\item This command saves the current Yabby session in the file
 SESSION\_NAME.tar.gz in the current directory.
\item Currently works only on system that have GNU tar command.
\end{enumerate}

% 4. Examples 

\item{Examples:}
\begin{enumerate}

\item
\begin{verbatim}
yabby> dump tmpsession

 Yabby session archived as 'tmpsession.tar.gz'
\end{verbatim}

\end{enumerate}

% 5. Requirements 

\item{Requirements:} {\em None}

% 6. Objects 

\item{Creates objects:} {\em None}

% 7. System scripts

\item{System scripts:} {\em None}

\end{description}

% ### END OF NEW COMMAND ####


% ### NEW COMMAND ###

\subsection[flush]{ \fbox{\tt flush} }

\index{\tt flush}

% Concise explanation

Deletes everything from the workspace.

% Command description

\begin{description}

% 1. Usage

\item{Usage:}

{\tt flush}

% 2. Options

\item{Options:}
\begin{description}
{\em None}
\end{description}

% 3. Notes

\item{Notes:}
\begin{enumerate}
{\em None}
\end{enumerate}

% 4. Examples 

\item{Examples:}
\begin{enumerate}

\item
\begin{verbatim}
yabby> flush

 [ workspace flushed ]
\end{verbatim}

\end{enumerate}

% 5. Requirements 

\item{Requirements:} {\em None}

% 6. Objects 

\item{Creates objects:} {\em None}

% 7. System scripts

\item{System scripts:} {\em None}

\end{description}

% ### END OF NEW COMMAND ####


% ### NEW COMMAND ###

\subsection[license]{ \fbox{\tt license} }

\index{\tt license}

% Concise explanation

Prints the Yabby license.

% Command description

\begin{description}

% 1. Usage

\item{Usage:}

{\tt license}

% 2. Options

\item{Options:}
\begin{description}
{\em None}
\end{description}

% 3. Notes

\item{Notes:}
\begin{enumerate}
{\em None}
\end{enumerate}

% 4. Examples 

\item{Examples:}
\begin{enumerate}

\item
\begin{verbatim}
yabby> license


                    GNU GENERAL PUBLIC LICENSE
                       Version 2, June 1991

 Copyright (C) 1989, 1991 Free Software Foundation, Inc.
                          675 Mass Ave, Cambridge, MA 02139, USA
 Everyone is permitted to copy and distribute verbatim copies
 of this license document, but changing it is not allowed.

                            Preamble

  The licenses for most software are designed to take away your
freedom to share and change it.  By contrast, the GNU General Public
....
\end{verbatim}

\end{enumerate}

% 5. Requirements 

\item{Requirements:} {\em None}

% 6. Objects 

\item{Creates objects:} {\em None}

% 7. System scripts

\item{System scripts:} {\em None}

\end{description}

% ### END OF NEW COMMAND ####

% ### NEW COMMAND ###

\subsection[help]{ \fbox{\tt help} }

\index{\tt help}

% Concise explanation

Prints help pages.

% Command description

\begin{description}

% 1. Usage

\item{Usage:}

{\tt help [ COMMAND ]}

% 2. Options

\item{Options:}
\begin{description}
{\em None}
\end{description}

% 3. Notes

\item{Notes:}
\begin{enumerate}
\item If no argument is given the list of all available commands will
be printed. If a command name is given the help about a particular
command will be printed. 
\end{enumerate}

% 4. Examples 

\item{Examples:}
\begin{enumerate}

\item
\begin{verbatim}
yabby> help quit

 Exits Yabby

\end{verbatim}

\end{enumerate}

% 5. Requirements 

\item{Requirements:} {\em None}

% 6. Objects 

\item{Creates objects:} {\em None}

% 7. System scripts

\item{System scripts:} {\em None}

\end{description}

% ### END OF NEW COMMAND ####


% ### NEW COMMAND ###

\subsection[print]{ \fbox{\tt print} }

\index{\tt print}

% Concise explanation

Prints the an object on the terminal screen or to a file.

% Command description

\begin{description}

% 1. Usage

\item{Usage:}

{\tt print [ options ] OBJ\_NAME.PROPERTY}

% 2. Options

\item{Options (general):}
\begin{description}
\item -f FILE\_NAME -- print to a file instead on the terminal screen.
\end{description}

\item{Options (seq objects only):}
\begin{description}
\item -l -- print the sequence as the three-letter code.
\item -c -- print the sequences as the CSV table.
\item -n N -- print N residues per line (both when printing one- and
 three-letter codes).
\item -t N -- truncate each sequence at N letters when writing
\end{description}

% 3. Notes

\item{Notes:}
\begin{enumerate}
\item Currently supported objects for printing are 'seq', 'motif',
and 'blastg'. 
\end{enumerate}

% 4. Examples 

\item{Examples:}
\begin{enumerate}

\item Print the sequence object tom20.seq
\begin{verbatim}
yabby> print tom20.seq

>A.thaliana [ A.thaliana ]
MDTETEFDRILLFEQIRQDAENTYKSNPLDADNLTRWGGVLLELSQFHSISDAKQMIQEA
ITKFEEALLIDPKKDEAVWCIGNAYTSFAFLTPDETEAKHNFDLATQFFQQAVDEQPDNT
HYLKSLEMTAKAPQLHAEAYKQGLGSQPMGRVEAPAPPSSKAVKNKKSSDAKYDAMGWVI
LAIGVVAWISFAKANVPVSPPR
>O.sativa [ O.sativa ]
MDMGAMSDPERMFFFDLACQNAKVTYEQNPHDADNLARWGGALLELSQMRNGPESLKCLE
DAESKLEEALKIDPMKADALWCLGNAQTSHGFFTSDTVKANEFFEKATQCFQKAVDVEPA
NDLYRKSLDLSSKAPELHMEIHRQMASQASQAASSTSNTRQSRKKKKDSDFWYDVFGWVV
LGVGMVVWVGLAKSNAPPQAPR
>L.esculentum [ L.esculentum ]
MDMQSDFDRLLFFEHARKTAETTYATDPLDAENLTRWAGALLELSQFQSVSESKKMISDA
ISKLEEALEVNPQKHDAIWCLGNAYTSHGFLNPDEDEAKIFFDKAAQCFQQAVDADPENE
LYQKSFEVSSKTSELHAQIHKQGPLQQAMGPGPSTTTSSTKGAKKKSSDLKYDVFGWVIL
AVGLVAWIGFAKSNMPXPAHPLPR
>S.tuberosum [ S.tuberosum ]
MEMQSEFDRLLFFEHARKSAETTYAQNPLDADNLTRWGGALLELSQFQPVAESKQMISDA
TSKLEEALTVNPEKHDALWCLGNAHTSHVFLTPDMDEAKVYFEKATQCFQQAFDADPSND
LYRKSLEVTAKAPELHMEIHRHGPMQQTMAAEPSTSTSTKSSKKTKSSDLKYDIFGWVIL
AVGIVAWVGFAKSNMPPPPPPPPQ
>P.taeda [ P.taeda ]
MEEMAIPQSDFDRLLFFEGARKAAENTYAVNPEDADNLTRWGGALLELSQFQQGPDCVKM
VKDAVSKLEEALKISPNKHDTLWCLGNAHTSHAFLIPEHEVAKIYFKMASKYFQQAVEQD
PTNELYRKSLELTEKAPELHLEVHKQILNPQSVAAGSSTVSNLKGSRKKKSSDLKYDIMG
WIVLAVGIAAWVGMAKSHVPPPPML
>G.max [ G.max ]
MDLQQSEFDRLLFFEHARKAAEAEYEKNPLDADNLTRWGGALLELSQFQSFPESKKMTQE
AVSKLEEALAVNPKKHDTLWCLGNAHTSQAFLIPDQEEAKVYFDKAAVYFQQAVDEDPSN
ELYRKSLEVAAKAPELHVEIHKQGFGQQQQAAATAGSSTSASTNTQKKKKSSDLKYDIFG
WIILAVGIVAWVGFAKSNLPPPPPPPPR
>Z.mays [ Z.mays ]
MEMGGMSDAERLFFFEMACKNSEVAYEQNPNDADNLTRWGGALLELSQVRTGPDSLKLLE
DAEAKLEEALQIDPNKSDALWCLGNAQTSHGFFTPDNAIANEFFTKATGCFQKAVDVEPA
NELYRKSLDLSMKAPELHLEIQRQMVSQAATQASSASNPRQSRKKKDNDFWYDVCGWVIL
GAGIVAWVGLARASMPPPTPPAR
\end{verbatim}

\item Print the sequence object tom20.seq to a file 'tom20.txt' 
\begin{verbatim}
yabby> print -f tom20.txt tom20.seq

 'tom20.seq' written to the file 'tom20.txt'
\end{verbatim}

\end{enumerate}

% 5. Requirements 

\item{Requirements:} The object to be printed must exist in the
workspace.

% 6. Objects 

\item{Creates objects:} {\em None}

% 7. System scripts

\item{System scripts:} \_print\_seq.pl, \_print\_motif.pl, \_print\_blastg.pl

\end{description}

% ### END OF NEW COMMAND ####

% ### NEW COMMAND ###

\subsection[restore]{ \fbox{\tt restore} }

\index{\tt restore}

% Concise explanation

Restores Yabby session saved with the command 'dump'

% Command description

\begin{description}

% 1. Usage

\item{Usage:}

{\tt restore SESSION\_NAME}

Where SESSION\_NAME is the archive name used when dumping
the session.

% 2. Options

\item{Options:}
\begin{description}
{\em None}
\end{description}

% 3. Notes

\item{Notes:}
\begin{enumerate}
\item The session will sored to a file named SESSION\_NAME.tar.gz.
\item Relies on GNU gar and gzip commands. These must be in the
 executable path.
\end{enumerate}

% 4. Examples 

\item{Examples:}
\begin{enumerate}

\item
\begin{verbatim}
yabby> restore tmpsession

 Yabby session 'tmpsession' restored
\end{verbatim}

\end{enumerate}

% 5. Requirements 

\item{Requirements:} {\em None}

% 6. Objects 

\item{Creates objects:} {\em None}

% 7. System scripts

\item{System scripts:} {\em None}

\end{description}

% ### END OF NEW COMMAND ####


% ### NEW COMMAND ###

\subsection[what]{ \fbox{\tt what} }

\index{\tt what}

% Concise explanation

Shows objects currently in the workspace.

% Command description

\begin{description}

% 1. Usage

\item{Usage:}

{\tt what}

% 2. Options

\item{Options:}
\begin{description}
{\em None}
\end{description}

% 3. Notes

\item{Notes:}
\begin{enumerate}
{\em None}
\end{enumerate}

% 4. Examples 

\item{Examples:}
\begin{enumerate}

\item
\begin{verbatim}
yabby> what

    objects        properties
  ------------------------------
    tom20          seq           

\end{verbatim}

\end{enumerate}

% 5. Requirements 

\item{Requirements:} {\em None}

% 6. Objects 

\item{Creates objects:} {\em None}

% 7. System scripts

\item{System scripts:} {\em None}

\end{description}

% ### END OF NEW COMMAND ####

% ##############################################################
\section{Sequence commands}
% ##############################################################

% ### NEW COMMAND ###

\subsection[seq\_comment]{ \fbox{\tt seq\_comment} }

\index{\tt seq\_comment}

% Concise explanation

Modifies sequence comments to add a unique number.

% Command description

\begin{description}

% 1. Usage

\item{Usage:}

{\tt seq\_comment OBJ\_NAME}

Where OBJ\_NAME is the internal Yabby name for the sequence(s).

% 2. Options

\item{Options:}
\begin{description}
{\em None}
\end{description}

% 3. Notes

\item{Notes:}
\begin{enumerate}
{\em None}
\end{enumerate}

% 4. Examples 

\item{Examples:}
\begin{enumerate}

\item
\begin{verbatim}
yabby> seq_comment tom20

 Comments modified.

\end{verbatim}

\end{enumerate}

% 5. Requirements 

\item{Requirements:} {\em None}

% 6. Objects 

\item{Creates objects:} {\em None}

% 7. System scripts

\item{System scripts:} {\em None}

\end{description}

% ### END OF NEW COMMAND ####


% ### NEW COMMAND ###

\subsection[seq\_load]{ \fbox{\tt seq\_load} }

\index{\tt seq\_load}

% Concise explanation

Loads sequence(s) from the database file.

% Command description

\begin{description}

% 1. Usage

\item{Usage:}

{\tt seq\_load [ options ] DBA\_FILE OBJ\_NAME}

Where DBA\_FILE is the name of the database file. OBJ\_NAME is
the internal Yabby name for the sequence(s).

% 2. Options

\item{Options:}
\begin{description}
\item -a -- append sequences to an already existing sequence
object OBJ\_NAME
\item -f -- the file format is FASTA (default)
\item -b -- the file format is BLOCKS
\end{description}

% 3. Notes

\item{Notes:}
\begin{enumerate}
\item Only BLOCKS format written by MEME \cite{meme} was tested.
\end{enumerate}

% 4. Examples 

\item{Examples:}
\begin{enumerate}

\item
\begin{verbatim}
yabby> seq_load tom20.seq tom20

 Reading the file 'tom20.seq' ..
 7 sequence(s) found.
\end{verbatim}

\end{enumerate}

% 5. Requirements 

\item{Requirements:} {\em None}

% 6. Objects 

\item{Creates objects:} seq

% 7. System scripts

\item{System scripts:} {\em None}

\end{description}

% ### END OF NEW COMMAND ####

