% chapter02.tex

 %%%%%%%%%%%%%%%%%%%%%%%%%%%%%%%%%%%%%%%%%%%%%%%%%%%%%%%%%%%%%%%%%%%%%%%%%%%%%
 %                                                                           %
 %    YABBY documentation                                                    %
 %    Copyright (C) 2007 Vladimir Likic                                      %
 %                                                                           %
 %    The files in this directory provided under the Creative Commons        %
 %    Attribution-NonCommercial-NoDerivs 2.1 Australia license               %
 %    http://creativecommons.org/licenses/by-nc-nd/2.1/au/                   %
 %    See the file license.txt                                               %
 %                                                                           %
 %%%%%%%%%%%%%%%%%%%%%%%%%%%%%%%%%%%%%%%%%%%%%%%%%%%%%%%%%%%%%%%%%%%%%%%%%%%%%

\setcounter{section}{0}

\chapter{Tutorial}

The data files used in this tutorial can be found in docs/data/ 
directory. Starting Yabby in this directory should allow one to
execute all examples given in the tutorial.

\section{Working with sequences}

\index{sequence files}

In Yabby sequences are represented as sequence objects. One way
to create a sequence object is to load sequences from a file.
Consider the file 'cad3.seq' which contains three sequences
from the Pfam CAD family in the FASTA format:

\begin{verbatim}
>Q53650_STAAU
YVATGIDYLVILILLFSQVKKGQVKHIWIGQYIGTAIVIGASLLVAQGVVNLIPQQWVIG
LLGLLPLYLGVKIWIKGEEDEDESSILSLFSSGKFNQLFLTMIFIVLASSADDFSIYIPY
FTTLSMSEIFIVTIVFLIMVGVLCYVSYRLASFDFISETIEKYERWIVPIVFIGLGIYIL
FENGTSNALISF
>Q97PJ0_STRPN
YISTSIDYLIILIILFAQLSQNKQKWHIYAGQYLGTGLLVGASLVAAYVVNFVPEEWMVG
LLGLIPIYLGIRFAIVGEDAEEEEEEIIERLEQSKANQLFWTVTLLTIASGGDNLGIYIP
YFASLDWSQTLVALLVFVIGIIIFCEISRVLSSIPLIFETIEKYERIIVPLVFILLGLYI
MYENGTIETFLIV
>P95773_STALU
YIAQALDLLVILLMFFARAKTRKEYRDIYIGQYVGSVALIVISLFFAFVLNYVPEKWILG
LLGLIPIYLGIKVAIYGDSDGEERAKKELNEKGLSKLVGTIAIVTIASCGADNIGLFVPY
FVTLSVTNLLITLFVFLILIFFLVFAAQKLANIPEVGEIVEKFGRWIMAVIYIALGLFII
IENDTIQTILGF
\end{verbatim}

\index{seq\_load}

To load this file in the workspace use the command 'seq\_load':

\begin{verbatim}
yabby> seq_load cad3.seq cad3

 Reading the file 'cad3.seq' ..
 3 sequence(s) found.

\end{verbatim}

Most Yabby commands which create a new object in the workspace
conform to the same pattern:

\begin{verbatim}
COMMAND [ options ] ARGS OBJ_NAME
\end{verbatim}

Where '[ options ]' is where flags and options are passed (if
any), ARGS are the arguments to the command, and OBJ\_NAME is
the name of the object to be created. In the example given
above, there were no options to the 'seq\_load' command; there
was one argument, the name of the file; and the object to be
created was named 'cad3'.

It is possible to inspect objects currently available in the
workspace:

\index{what}

\begin{verbatim}
yabby> what

    object(s)      type
  ------------------------------
    cad3           seq           

\end{verbatim}

This listing shows that one object is currently in the workspace,
of the 'seq' type. It is possible to load the same sequences
under a different name: 

\begin{verbatim}
yabby> seq_load cad3.seq cad3_second

 Reading the file 'cad3.seq' ..
 3 sequence(s) found.

yabby> what

    object(s)      type
  ------------------------------
    cad3           seq           
    cad3_second    seq           

\end{verbatim}

The two objects 'cad3.seq' and 'cad3\_second.seq' are identical.

\index{print}

The command 'print' allows one to print the sequence object:

\begin{verbatim}
yabby> print cad3.seq

>Q53650_STAAU [ Q53650_STAAU ]
YVATGIDYLVILILLFSQVKKGQVKHIWIGQYIGTAIVIGASLLVAQGVVNLIPQQWVIG
LLGLLPLYLGVKIWIKGEEDEDESSILSLFSSGKFNQLFLTMIFIVLASSADDFSIYIPY
FTTLSMSEIFIVTIVFLIMVGVLCYVSYRLASFDFISETIEKYERWIVPIVFIGLGIYIL
FENGTSNALISF
>Q97PJ0_STRPN [ Q97PJ0_STRPN ]
YISTSIDYLIILIILFAQLSQNKQKWHIYAGQYLGTGLLVGASLVAAYVVNFVPEEWMVG
LLGLIPIYLGIRFAIVGEDAEEEEEEIIERLEQSKANQLFWTVTLLTIASGGDNLGIYIP
YFASLDWSQTLVALLVFVIGIIIFCEISRVLSSIPLIFETIEKYERIIVPLVFILLGLYI
MYENGTIETFLIV
>P95773_STALU [ P95773_STALU ]
YIAQALDLLVILLMFFARAKTRKEYRDIYIGQYVGSVALIVISLFFAFVLNYVPEKWILG
LLGLIPIYLGIKVAIYGDSDGEERAKKELNEKGLSKLVGTIAIVTIASCGADNIGLFVPY
FVTLSVTNLLITLFVFLILIFFLVFAAQKLANIPEVGEIVEKFGRWIMAVIYIALGLFII
IENDTIQTILGF
\end{verbatim}

Upon reading the sequences Yabby has taken the first string
in the FASTA comment to be the sequence ID, and the full
comment is re-inserted within the square brackets. In this
case the only comment was the ID string, and this is merely
repeated within the square brackets. 

The command 'print' can send sequences to a file, instead
of printing them in the terminal window:

\begin{verbatim}
yabby> print -f tmp.fasta cad3.seq

 'cad3.seq' written to the file 'tmp.fasta'

\end{verbatim}


