% chapter01.tex

 %%%%%%%%%%%%%%%%%%%%%%%%%%%%%%%%%%%%%%%%%%%%%%%%%%%%%%%%%%%%%%%%%%%%%%%%%%%%%
 %                                                                           %
 %    YABBY documentation                                                    %
 %    Copyright (C) 2007 Vladimir Likic                                      %
 %                                                                           %
 %    The files in this directory provided under the Creative Commons        %
 %    Attribution-NonCommercial-NoDerivs 2.1 Australia license               %
 %    http://creativecommons.org/licenses/by-nc-nd/2.1/au/                   %
 %    See the file license.txt                                               %
 %                                                                           %
 %%%%%%%%%%%%%%%%%%%%%%%%%%%%%%%%%%%%%%%%%%%%%%%%%%%%%%%%%%%%%%%%%%%%%%%%%%%%%

\setcounter{section}{0}

\chapter{Introduction}

\section{What is Yabby}

- Introduction
- Motivation
- Features
- License

Yabby is released as open source, under the GNU Public License
version 2.

\section{Installation}

\subsection{Checking Perl}

\index{installation}
\index{Perl}
\index{system requirements}

Yabby has been developed with Perl version 5.X, a freely available
general purpose scripting language. Perl stands for "Practical
Extraction and Report Language" and is particularly well suited for
the manipulation of data stored in plain text files. This seems
all too often to be required in bioinformatics applications.

Before attempting the installation, it is highly recommended to
check if the Perl interpreter is present your computer system.
For example, on a Linux system,

\begin{verbatim}
$ perl -v

This is perl, v5.8.5 built for i386-linux-thread-multi

Copyright 1987-2004, Larry Wall

Perl may be copied only under the terms of either the Artistic License or the
GNU General Public License, which may be found in the Perl 5 source kit.

Complete documentation for Perl, including FAQ lists, should be found on
this system using `man perl' or `perldoc perl'.  If you have access to the
Internet, point your browser at http://www.perl.com/, the Perl Home Page.
\end{verbatim}

Perl 5 and later is required for Yabby.  The next step is to find where
exactly is the Perl interpreter located, as this information will be
required for Yabby installation:

\begin{verbatim}
$ which perl
/usr/local/bin/perl
\end{verbatim}

\subsection{Downloading Yabby}

Yabby source code can be browsed from the Google Code servers, at
the URL: http://code.google.com/p/yabby/. Under the
section "Source" one can find the instructions for downloading the
source code. The same page provides the link under "This project's
Subversion repository can be viewed in your web browser" which allows
one to browse the source code on the server without actually
downloading it.

Google servers maintain the source code by the program called 'subversion'
(an open-source version control system).  To download the source code
one needs to use the subversion client program called 'svn'.  The 'svn'
client exists for all mainstream operating systems\footnote{For example,
on Linux CentOS 4 the RPM package 'subversion-1.3.2-1.rhel4.i386.rpm'
provides the subversion client 'svn'.}, for more information see
http://subversion.tigris.org/.  The book about subversion is freely
available on-line at http://svnbook.red-bean.com/. Subversion has
extensive functionality however only the very basic functionality
is needed to download Yabby.  Assuming that the computer is connected
to the internet, the following command will download the latest Yabby 
source code in the current directory:

\begin{verbatim}
$ svn checkout http://yabby.googlecode.com/svn/trunk/ yabby
A    yabby/yabby.pl
A    yabby/LICENSE
A    yabby/lib
A    yabby/lib/blast.pl
A    yabby/lib/hmm_score2seq.pl
A    yabby/lib/seq_strip.pl
A    yabby/lib/seq_comment.pl
A    yabby/lib/hmm_score.pl
A    yabby/lib/blastg.pl
A    yabby/lib/motif_cmp.pl
A    yabby/lib/seq_unique.pl
A    yabby/lib/yabby_seq.pm
....further output deleted....
\end{verbatim}

\subsection{Installing Yabby}

Yabby installation requires that the file 'yabby/lib/yabby.pl' is
modified to to set the path to the Perl language interpreter,
and to Yabby libraries.

If Yabby code was downloaded in the directory /home/jake/ (and
therefore the script 'yabby.pl' is in /home/jake/yabby/), to set
the path to Yabby libraries the following two lines need to be
set: 

\begin{verbatim}
use lib "/home/jake/yabby/lib";
$LIB_DIR = "/home/jake/yabby/lib";
\end{verbatim}

The path to the Perl interpreter is set in the first line of the
file 'yabby.pl':

\begin{verbatim}
#!/usr/bin/perl
\end{verbatim}

The script yabby.pl needs to have executable permissions:

\begin{verbatim}
$ chmod +x yabby.pl
\end{verbatim}

If this is all set, running the script 'yabby.pl' will start Yabby:

\begin{verbatim}
$ yabby.pl

 - YABBY version 0.1 - 
   Copyright (c) 2004-7 Vladimir Likic
 [ 35 command(s) ready ]

yabby>
\end{verbatim}

It is often useful to create a sumbolic link in a directory which
is included the PATH variable, such as /usr/local/bin or ~/bin:

\begin{verbatim}
ln -s /home/jake/yabby/yabby.pl /home/jake/bin/yabby
\end{verbatim}

This would allow Yabby to be run from any directory, simply by
typing 'yabby'.

\section{Running Yabby}

\subsection{Running interactive session}

Yabby can be run interactively or from the command script. To
start an Yabby interactive session one needs to start the Yabby
interface from the Unix shell. Here is the simplest Yabby session:

\begin{verbatim}
$ yabby

 - YABBY version 0.1 - 
   Copyright (c) 2004-7 Vladimir Likic
 [ 35 command(s) ready ]

yabby> quit

 bye-bye
\end{verbatim}

\subsection{Yabby command scripts}

In order to run Yabby from the command script, the command file
needs to be prepared first. Such a file lists Yabby commands
one per line, with optional blank lines (lines which start with
the \% character are ignored). For example, the following input
file, named test.yab,

\begin{verbatim}
% test.yab -- test input script

seq_load cad3.seq cad3
seq_info -l cad3
\end{verbatim}

could be run with the Unix shell input redirection:

\begin{verbatim}
$ yabby < test.yab

 - YABBY version 0.1 - 
   Copyright (c) 2004-7 Vladimir Likic
 [ 35 command(s) ready ]

yabby> % test.yab -- test input script
yabby> yabby> 
 Reading the file 'cad3.seq' ..
 3 sequence(s) found.

yabby> 
 'cad3' contains 3 sequence(s)
  1 -> Q53650_STAAU, 192 residues
  2 -> Q97PJ0_STRPN, 193 residues
  3 -> P95773_STALU, 192 residues

yabby> yabby> 
 bye-bye
\end{verbatim}

When run from the command script the actual commands are not
echoed back, only the command's screen output as well as the
comments.

\subsection{Executing Unix commands}

\index{unix commands}

Any command which is not recognized by Yabby is assumed to be
an Unix command, and Yabby will attempt to execute it. Consider
the following example:

\begin{verbatim}
yabby> ls
1BT0.pdb  cox1.seq     LmjFmockup.pep  needle.out
cad3.seq  dna.seq      m2.blocks       README
cad.seq   hmmpfam.out  meme.out        test.yab
yabby> l

 [ UNIX command 'l' failed ]
\end{verbatim}

Because there is no Yabby command {\tt ls}, it was assumed to
be a system command and executed.  The output was printed on
the screen, listing the files and directories in the directory
where Yabby was started.

Subsequently, the command {\tt l} was given but failed because
there is no such Yabby or Unix command. If {\tt l} was an alias
to something (say {\tt ls -CF}) the command would fail regardless,
because Yabby does not know about shell aliases.

There are no inherent limitations to which Unix commands can be
executed within Yabby. It is possible to run a text editor, such
as "vi" (and then simply resume the Yabby session after exiting
the editor), or even start programs with GUI such as "gnuplot",
or a Unix terminal window.

A subtle but important point is that in Yabby Unix commands are
not executed through the Unix shell.  The consequence of this
is that the (sometimes important) functions provided the the
Unix shell, such as file globbing, are not available. For example:

\begin{verbatim}
yabby> ls *
ls: *: No such file or directory

 [ UNIX command 'ls *' failed ]
\end{verbatim}

A handy trick which allows one to go about Unix business is
to temporarily suspend Yabby. This is actually a feature provided
by some unix shells (in combination with the system's terminal driver),
and has little to do with Yabby. In short, typing Ctrl-Z within
Yabby will suspend the current Yabby session, and return user
to the Unix shell. Issuing the command "fg" to the same shell
will return the suspended Yabby session:

\begin{verbatim}
yabby> [Cntrl-Z]
[1]+  Stopped                 yabby
$ ls -CF
1BT0.pdb  cox1.seq     LmjFmockup.pep  needle.out
cad3.seq  dna.seq      m2.blocks       README
cad.seq   hmmpfam.out  meme.out        test.yab
$ fg

yabby>
\end{verbatim}

Ctrl-Z was typed on the first line, which was not echoed back. Note
that the second command prompt (starting with the {\tt \$} character)
is the Unix shell prompt, and typing "fg" had returned user to
the suspended Yabby session.

