% chapter01.tex

 %%%%%%%%%%%%%%%%%%%%%%%%%%%%%%%%%%%%%%%%%%%%%%%%%%%%%%%%%%%%%%%%%%%%%%%%%%%%%
 %                                                                           %
 %    YABBY documentation                                                    %
 %    Copyright (C) 2007 Vladimir Likic                                      %
 %                                                                           %
 %    The files in this directory provided under the Creative Commons        %
 %    Attribution-NonCommercial-NoDerivs 2.1 Australia license               %
 %    http://creativecommons.org/licenses/by-nc-nd/2.1/au/                   %
 %    See the file license.txt                                               %
 %                                                                           %
 %%%%%%%%%%%%%%%%%%%%%%%%%%%%%%%%%%%%%%%%%%%%%%%%%%%%%%%%%%%%%%%%%%%%%%%%%%%%%

\setcounter{section}{0}

\chapter{Introduction}

\section{What is Yabby}

Yabby is released as open source, under the GNU Public License
version 2.

\section{Installation}

\subsection{Checking Perl}

\index{installation}
\index{Perl}
\index{system requirements}

Yabby has been developed with Perl version 5.X, a freely available
general purpose scripting language. Perl stands for "Practical
Extraction and Report Language" and is particularly well suited for
the manipulation of data stored in plain text files. This seems
all too often to be required in bioinformatics applications.

Before attempting the installation, it is highly recommended to
check if the Perl interpreter is present your computer system.
For example, on a Linux system,

\begin{verbatim}
$ perl -v

This is perl, v5.8.5 built for i386-linux-thread-multi

Copyright 1987-2004, Larry Wall

Perl may be copied only under the terms of either the Artistic License or the
GNU General Public License, which may be found in the Perl 5 source kit.

Complete documentation for Perl, including FAQ lists, should be found on
this system using `man perl' or `perldoc perl'.  If you have access to the
Internet, point your browser at http://www.perl.com/, the Perl Home Page.
\end{verbatim}

Perl 5 and later is required for Yabby.  The next step is to find where
exactly is the Perl interpreter located, as this information will be
required for Yabby installation:

\begin{verbatim}
$ which perl
/usr/local/bin/perl
\end{verbatim}

\subsection{Downloading Yabby}

Yabby source code can be browsed from the Google Code servers, at
the following URL: http://code.google.com/p/yabby/. Under the
section "Source" one can find the instructions for downloading the
source code. The same page provides the link under "This project's
Subversion repository can be viewed in your web browser" which allows
one to browse the source code on the server without actually
downloading it.

The Yabby source code can be downloaded from the Google Code directory.
Google servers maintain the source code by the program called 'subversion'
(an open-source version control system).  To download the source code
one needs to use the subversion client program called 'svn'.  The 'svn'
client exists for all mainstream operating systems\footnote{For example,
on Linux CentOS 4 the RPM package 'subversion-1.3.2-1.rhel4.i386.rpm'
provides the subversion client 'svn'.}, for more information see
http://subversion.tigris.org/.  The book about subversion is freely
available on-line at http://svnbook.red-bean.com/. Subversion has
extensive functionality however only the very basic functionality
is needed to download Yabby.  Assuming that the computer is connected
to the internet, the following command will download the latest Yabby 
source code in the current directory:

\begin{verbatim}
]$ svn checkout http://yabby.googlecode.com/svn/trunk/ yabby
A    yabby/yabby.pl
A    yabby/LICENSE
A    yabby/lib
A    yabby/lib/blast.pl
A    yabby/lib/hmm_score2seq.pl
A    yabby/lib/seq_strip.pl
A    yabby/lib/seq_comment.pl
A    yabby/lib/hmm_score.pl
A    yabby/lib/blastg.pl
A    yabby/lib/motif_cmp.pl
A    yabby/lib/seq_unique.pl
A    yabby/lib/yabby_seq.pm
....further output deleted....
\end{verbatim}

\subsection{Installing Yabby}

Yabby installation requires that the file 'yabby/lib/yabby.pl' is
modified to to set the path to the Perl language interpreter,
and to Yabby libraries.

If Yabby code was downloaded in the directory /home/jake/ (and
therefore the script 'yabby.pl' is in /home/jake/yabby/), to set
the path to Yabby libraries the following two lines need to be
set: 

\begin{verbatim}
use lib "/home/jake/yabby/lib";
$LIB_DIR = "/home/jake/yabby/lib";
\end{verbatim}

The path to the Perl interpreter is set in the first line of the
file 'yabby.pl':

\begin{verbatim}
#!/usr/bin/perl
\end{verbatim}

The script yabby.pl needs to have executable permissions:

\begin{verbatim}
$ chmod +x yabby.pl
\end{verbatim}

If this is all set, running the script 'yabby.pl' will start Yabby:

\begin{verbatim}
]$ yabby.pl

 - YABBY version 0.1 - 
   Copyright (c) 2004-7 Vladimir Likic
 [ 35 command(s) ready ]

yabby>
\end{verbatim}

If is 

